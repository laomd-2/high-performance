\documentclass[UTF8,a4paper,12pt]{article}

\usepackage[version=3]{mhchem} % Package for chemical equation typesetting
\usepackage{ctex}
\usepackage{siunitx} % Provides the \SI{}{} and \si{} command for typesetting SI units
\usepackage{graphicx} % Required for the inclusion of images
\usepackage{subfigure}
\usepackage{natbib} % Required to change bibliography style to APA
\usepackage{amsmath} % Required for some math elements
\usepackage{enumitem}
\usepackage{indentfirst}

\usepackage[top=2cm, bottom=2cm, left=2cm, right=2cm]{geometry}
\usepackage{algorithm}
\usepackage{algorithmicx}
\usepackage{algpseudocode}

\renewcommand{\labelenumi}{\alph{enumi}.} % Make numbering in the enumerate environment by letter rather than number (e.g. section 6)
\floatname{algorithm}{算法}
\renewcommand{\algorithmicrequire}{\textbf{输入:}}
\renewcommand{\algorithmicensure}{\textbf{输出:}}

\usepackage{listings}
% \usepackage{times} % Uncomment to use the Times New Roman font
%----------------------------------------------------------------------------------------
%	DOCUMENT INFORMATION
%---------------------------------------------------------------------------------------

\begin{document}

\begin{titlepage}
    \begin{center}
        \phantom{Start!}
    	  \vspace{2cm}
        \center{\zihao{1} 《高性能程序设计基础》}
        \center{\zihao{1} 实验报告}
        {
            \setlength{\baselineskip}{40pt}
            \vspace{1cm}
            \zihao{-2}
            \center{
                \begin{tabular}{cc}
                  实验序号:& \underline{~~~~~~实验三~~~~~~}\\
                  实验名称:& \underline{~MPI矩阵向量乘法的并行算法~}\\
                  姓\qquad 名:& \underline{~~~~~~劳马东~~~~~~}  \\
              	  学\qquad 号:& \underline{~~~~~16337113~~~~~}  \\
              	\end{tabular}
            }
        }
    \end{center}
\end{titlepage}

\section{实验题目}
\begin{enumerate}[itemindent=0.5em,label=(\arabic*)]
  \item 用MPI完成稠密矩阵向量乘法的并行算法
  \begin{enumerate}
    \item 任务划分按照数据划分方法
    \begin{enumerate}
      \item 按输出数据划分
      \item 按输入数据划分
    \end{enumerate}
    \item 矩阵和向量从磁盘读入
	  \item 结果输出到磁盘
    \item 矩阵和向量文件格式统一按三元组存放
    \item 矩阵很大,可能一个节点存不下
  \end{enumerate}
  \item 用MPI完成稀疏矩阵向量乘法的并行算法
  \begin{enumerate}
    \item 根据非零元分布划分矩阵
  \end{enumerate}
\end{enumerate}
\section{实验目的}
\begin{enumerate}[itemindent=0.5em,label=(\arabic*)]
  \item 熟悉MPI全局聚集函数的使用;
  \item 掌握常见任务划分方法。
\end{enumerate}

\section{实验要求}
\begin{enumerate}[itemindent=0.5em,label=(\arabic*)]
  \item 计算算法的加速比,并列表
\end{enumerate}
\section{实验过程}
\subsection{稠密矩阵向量乘法}
\begin{enumerate}[itemindent=0.5em,label=\arabic*、]
  \item 矩阵划分
  \begin{enumerate}
    \item 按行划分
    \par 输入文件的矩阵格式是按列给出,因此根进程每次读入一列,然后将这一列分发到各个进程。
    由于一列的大小不一定是进程数的整数倍,因此分发不是平均的,各个进程负载大小(行数)通过
    一个行负载计算模块得到。收到的数据需要经过一次矩阵变换——转置,因为一列在物理上存储为一维数组,本质上是一个
    行向量,而收到的数据逻辑上是列向量。如图1。
    \begin{figure}[h]
    \begin{center}
    \includegraphics[width=0.8\textwidth]{divide_on_row.png} % Include the image placeholder.png
    \caption{行划分流程}
    \end{center}
    \end{figure}
    \newpage
    \par 图2、3、4、5是该过程的主要代码。$global\_one\_col$是一个一维数组,一行结束标志为文件末尾或读入的列号为下一列;
    细节之处在于如果是由于读入了下一列而中止循环,那么下一列的第一行已经被读走了;一列的分发是异步的,因为0号进程之后不需要
    用到这次读的一列,如此可以提高性能。
    \begin{figure}[h]
    \begin{center}
    \includegraphics[width=0.6\textwidth]{divide_on_row2_1.png} % Include the image placeholder.png
    \caption{读入一行}
    \end{center}
    \end{figure}

    \begin{figure}[h]
    \begin{center}
    \includegraphics[height=0.15\textwidth]{divide_on_row2_4.png} % Include the image placeholder.png
    \caption{细节处理}
    \end{center}
    \end{figure}

    \begin{figure}[h]
    \begin{center}
    \includegraphics[width=0.8\textwidth]{divide_on_row2_2.png} % Include the image placeholder.png
    \caption{分发}
    \end{center}
    \end{figure}

    \begin{figure}[h]
    \begin{center}
    \includegraphics[width=0.6\textwidth]{divide_on_row2_3.png} % Include the image placeholder.png
    \caption{矩阵转置}
    \end{center}
    \end{figure}
\newpage
  \end{enumerate}
  \item 局部稠密矩阵与向量相乘
  \par 由于本地矩阵的每一列都是完整的,因此该算法就是普通的矩阵-向量乘法。
  \par 需要强调的是,向量x也按行划分的方式
  分布在各个进程之中,依然是为了性能。
  假设各个进程接收字节的速率为:
  \begin{equation}
    d_{min} = \min\{{d_0, d_1, ..., d_i, ..., d_{N-1}\}}
  \end{equation}
  无论如何每个进程在做矩阵-向量乘法的时候都需要得到完整的x。假如
  0号进程读取向量并用广播的方式发送给其他进程。
  假设进程数为N,向量字节数为F,那么该过程至少需要的时间为:
  \begin{equation}
    t_{broadcast} = \frac{N \times F}{d_{min}} = O(N)
  \end{equation}
  而采用先分发再蝶形收集的方法,至少需要的时间为:
  \begin{equation}
    \begin{aligned}
      t_2 = t_{scatter} + t_{all\_gather} =
      \sum_{i=1}^{N}\frac{\frac{F}{N}}{d_i} + \sum_{i=1}^{\log_2N}{\frac{i \times F}{d_{min}}}
      = O((\log_2N)^2)
    \end{aligned}
  \end{equation}
  此外,分发过程与0号进程读向量的过程可以并发,进一步
  提高性能。
  \begin{figure}[h]
  \begin{center}
  \includegraphics[width=0.75\textwidth]{d_mat_vec_mul.png} % Include the image placeholder.png
  \caption{向量聚集与乘法}
  \end{center}
  \end{figure}

  \item 局部结果聚合
  \par 使用MPI的gatherv函数收集每个进程本地的y向量即可。
\end{enumerate}
\subsection{稀疏矩阵向量乘法}
\begin{enumerate}[itemindent=0.5em,label=\arabic*、]
  \item 矩阵划分——按元素划分
  \par 简单起见,首先读入完整的矩阵,然后调用MPI的scatterv函数按元素顺序分发。效率更高
  的方法是根据每个进程的负载读入相应个数的元素,然后异步send到对应进程。但由于在实验中发现
  读取整个矩阵的时间极短(不足1秒),因此采用简单方法。
  \begin{figure}[h]
  \begin{center}
  \includegraphics[width=0.8\textwidth]{divide_on_elem.png} % Include the image placeholder.png
  \caption{按元素划分的两种方法}
  \end{center}
  \end{figure}
  \item 局部矩阵与向量相乘
  \par 遍历局部矩阵中的每个元素,获得其行列下标,与全局向量x对应位置相乘求和,放入局部结果矩阵$local\_y$(与全局y大小相同)即可。
  \begin{figure}[h]
  \begin{center}
  \includegraphics[width=0.8\textwidth]{s_mat_vec_mul.png} % Include the image placeholder.png
  \caption{稀疏矩阵与向量乘法}
  \end{center}
  \end{figure}
  \item 局部结果聚合
  \par 每个进程本地的$local\_y$向量相加就是全局y,因此调用MPI的reduce函数求和即可。
\end{enumerate}

\section{实验结果及分析}
\subsection{正确性验证}
测试进程数为4,测试矩阵与向量为:
\begin{equation}
  A =
 \begin{bmatrix}
   2 & 3 & 5 & 2 & 1 & 1\\
   1 & 0 & 3 & 4 & 5 & 0\\
   5 & 3 & 5 & 4 & 1 & 0\\
   0 & 0 & 3 & 5 & 5 & 4\\
   2 & 5 & 5 & 4 & 0 & 5\\
   3 & 4 & 2 & 1 & 4 & 3
  \end{bmatrix} \qquad
  x =
  \begin{bmatrix}
    3\\
    4\\
    4\\
    1\\
    5\\
    4
    \end{bmatrix}
\end{equation}
\begin{enumerate}[itemindent=0.5em,label=\arabic*、]
  \item 稠密矩阵
  \par 总共6行,进程0分到前两行(2 3 5 2 1 1 1 3 4 5,只输出非0元素),依此类推,结果正确;
  \par 进程0本地y为$(49=2*3+3*4+5*4+2*1+1*5+1*4,44)$,依此类推,结果正确;
  \par 全局y为$(49,44,56,58,70,66)$,正确。
  \begin{figure}[H]
  \centering
  \subfigure[行划分结果]{
  \includegraphics[width=0.3\textwidth]{result_row_divide.png}}
  \subfigure[进程本地y]{
  \includegraphics[width=0.3\textwidth]{result_local_y1.png}}
  \subfigure[全局y]{
  \includegraphics[width=0.3\textwidth]{result_y1.png}}
  \end{figure}

  \item 稀疏矩阵
  \par 总共30个元素,进程0和1按顺序分到8个,进程2、3按顺序分到7个,正确;
  \par 进程0本地y为$(18=2*3+3*4,3,27,26,9)$,以此类推,结果正确;
  \par 全局y正确。

  \begin{figure}[H]
  \centering
  \subfigure[行划分结果]{
  \includegraphics[width=0.3\textwidth]{result_row_elem.png}}
  \subfigure[进程本地y]{
  \includegraphics[width=0.3\textwidth]{result_local_y2.png}}
  \subfigure[全局y]{
  \includegraphics[width=0.3\textwidth]{result_y2.png}}
  \end{figure}
\end{enumerate}

\subsection{加速比计算}
\begin{enumerate}[itemindent=0.5em,label=\arabic*、]
  \item 稠密矩阵
  \begin{center}
      \begin{tabular}{cccc}
      \hline
      进程数 & 运行时间/秒 & 加速比 & 效率/\%\\
      \hline
      1	& 122 & 1 & 100\\
      2	& 70.339 & 1.73 & 86.72\\
      4	& 41.676 & 2.92 & 73.18\\
      8	& 25.574 & 4.77 & 59.63\\
      16 & 17.47 & 6.98 & 43.64\\
      28 & 14.274 & 8.55 & 30.52\\
      \hline
      \end{tabular}
  \end{center}

  \begin{figure}[H]
  \centering
  \subfigure[时间]{
  \includegraphics[width=0.3\textwidth]{time_dense.png}}
  \subfigure[加速比]{
  \includegraphics[width=0.3\textwidth]{speedup_dense.png}}
  \subfigure[效率]{
  \includegraphics[width=0.3\textwidth]{eff_dense.png}}
  \end{figure}

  \item 稀疏矩阵
  \begin{center}
      \begin{tabular}{cccc}
      \hline
      进程数 & 运行时间/秒 & 加速比 & 效率/\%\\
      \hline
      1 &	1.226	&	1	&	100\\
      2 &	1.216	&	1.008	&	50.41\\
      4 &	1.223	&	1.002	&	25.06\\
      8	&	1.278	&	0.959	&	11.99\\
      16 & 1.438 & 0.852 & 5.33\\
      28 & 1.467 & 0.836 & 2.98\\
      \hline
      \end{tabular}
  \end{center}

  \begin{figure}[H]
  \centering
  \subfigure[时间]{
  \includegraphics[width=0.3\textwidth]{time_sparse.png}}
  \subfigure[加速比]{
  \includegraphics[width=0.3\textwidth]{speedup_sparse.png}}
  \subfigure[效率]{
  \includegraphics[width=0.3\textwidth]{eff_sparse.png}}
  \end{figure}

\end{enumerate}
\end{document}
